
\chapter{Introduction}
This document is structured according to the IEEE standard 830-1998 Recommended Practice for Software Requirements Specifications.

 
\section{Purpose}
The purpose of this Software Requirements Specifications document is to have a central point for all the requirements of the bookkeeping system. It is not written with the goal to have a document with definite and final requirements for this system. During the process of developing this system requirements will be added and modified.

\subsection{Vision}
Provide unified bookkeeping experience for operations, run catalogue and management.


\section{Scope of the Bookkeeping System}
The scope of the project to develop the bookkeeping system is restricted to keeping track of the configuration of ALICE, data produced by ALICE, and computations on this data. The bookkeeping system is not a monitoring system for ALICE.

\section{Definitions, Acronyms, and Abbreviations}

\subsection{Log Entry}
A Log Entry is a text message that describe an intervention or an event that happened. It can be generated either by humans (e.g. a shifter enters his/her end-of-shift report) or by machines (e.g. a detector logs some abnormal situation automatically). 

\subsection{Run}
A Run is a unique ID that identifies a synchronous data processing session in the $O^2$ computing system with a specific and well-defined configuration. It normally ranges from a few minutes to tens of hours. It is generated and managed by the $O^2$ system. 

\subsection{LHC Fill}
An LHC Fill is a unique ID that identifies a period of activity in the LHC accelerator. It normally ranges from a few minutes to tens of hours. It is generated and managed by the LHC system and published via DIP protocol. 

\begin{longtable}
  \caption{Acronyms}
  \label{tab:acronyms}

  \begin{center}
    \begin{tabular}{ll}
    ALICE & A Large Ion Collider Experiment\\
    API & Application Programming Interface\\
    DAQ & Data Acquisition subsystem \\
    ID & Identity Document\\
    IEEE & Institute of Electrical and Electronics Engineers\\
     LHC  & Large Hydron Collider\\\
     $O^2$ & Online and Offline\\
     
       & \\
    \end{tabular}
  \end{center}
\end{longtable}




\section{References}

\section{Overview}
